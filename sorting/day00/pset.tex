
\documentclass{article}
\usepackage[utf8]{inputenc}

\title{\large{\textsc{In-Class 5: Sorting}}}
\date{}

\usepackage{natbib}
\usepackage{graphicx}
\usepackage{amsmath}
\usepackage{amsfonts}
\usepackage{mathtools}
\usepackage[a4paper, portrait, margin=0.8in]{geometry}

\usepackage{listings}


\newcommand\perm[2][n]{\prescript{#1\mkern-2.5mu}{}P_{#2}}
\newcommand\comb[2][n]{\prescript{#1\mkern-0.5mu}{}C_{#2}}
\newcommand*{\field}[1]{\mathbb{#1}}

\DeclarePairedDelimiter\ceil{\lceil}{\rceil}
\DeclarePairedDelimiter\floor{\lfloor}{\rfloor}

\newcommand{\Mod}[1]{\ (\text{mod}\ #1)}

\begin{document}

\maketitle

\subsection*{}

\begin{enumerate}

\item Given an array, return true if the array has a duplicate element in it.

\item Suppose you have an array A[] of N distinct items which is nearly sorted: each item at most k positions away from its position in the sorted order. Design an algorithm to sort the array in O(Nlogk) time.
	
%%%%% PROBLEM 1 %%%%%
\item Given a list of \texttt{Interval}s as defined in the \texttt{Interval} class below, merge overlapping \texttt{Interval}s and output the resulting list of mutually exclusive \texttt{Interval}s.  For example:

\begin{itemize}
    \item Given the list of \texttt{Interval}s \texttt{[(2,4), (5,7), (1,3), (6,8)]}, the  intervals \texttt{(1,3)} and \texttt{(2,4)} overlap with each other, so they should be merged and become \texttt{(1,4)}. \texttt{(5,7)} and \texttt{(6,8)} also overlap, creating \texttt{(5,8)}.  The final solution for this list of \texttt{Interval}s would be \texttt{[(1,4), (5,8)]}.
\end{itemize}

The definition for the Intervals class is as follows:

\begin{lstlisting}[language=Java]
 public class Interval {
    int start;
    int end;
    Interval(int s, int e) { start = s; end = e; }
 }
\end{lstlisting}
    
%%%%% PROBLEM 2 %%%%%
\item Given the head of a linked list, sort it in place using O($n$log$n$) time and O(logN) space.
    
\end{enumerate}

\clearpage

\begin{lstlisting}[language=Python]
def contains_duplicate(A):
    # O(N) time, O(N) space
    s = set()
    for i in A:
        if i in s:
            return False
        s.add(i)

# A is sorted from [lo, mid) and [mid, hi). Sorts elements from [lo, hi)
def merge(A, lo, mid, hi):
    i = lo
    j = mid
    B = []
    while (i < mid) or (j < hi):
        if (i == mid) or (j < hi and (A[j] < A[i])):
            B.append(A[j])
            j += 1
        else:
            B.append(A[i])
            i += 1
    for k in range(lo, hi):
        A[k] = B[k - lo]

# Sort element in [lo, hi)
def mergeSort(A, lo, hi):
    if hi - lo < 2:
        return
    mid = (hi + lo) / 2
    mergeSort(A, lo, mid)
    mergeSort(A, mid, hi)
    merge(A, lo, mid, hi)


def sortK(A, k):
    # first, sort from [0, 2k). After this, first k smallest elements are in
    # the correct position. then sort from [k, 3k). Now, the first 2k smallest
    # elements are in the correct position. Continue across the entire array.
    # This inner loop runs (N/k) times. Each loop iteration performs a
    # mergesort on a length-k array, which is complexity klogk. Therefore,
    # overall complexity is Nlogk.
    for i in range(0, len(A), k):
        mergeSort(A, i, min(i + 2*k, len(A)))
\end{lstlisting}

\begin{lstlisting}[language=Java]
MERGE_INTERVALS(intervals) {
    intervals.sortByStartTime();
    
    // use an iterator here
    for (interval i : intervals) {
        while (i.end > i.next.start) {
            // merge sets i.end to max(i.end, i.next.end)
            merge(i, i.next);
        }
    }
}

SORT_LINKED_LIST(linkedlist) {
    leftList, rightList = split(linkedlist)
    SORT_LINKED_LIST(leftList)
    SORT_LINKED_LIST(rightList)
    merge(leftList, rightList)
    }
}

\end{lstlisting}

\end{document}
