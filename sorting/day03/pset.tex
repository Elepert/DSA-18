\documentclass{article}
\usepackage[utf8]{inputenc}

\title{\large{\textsc{In-Class 8: Counting Sort}}}
\date{}

\usepackage{natbib}
\usepackage{graphicx}
\usepackage{amsmath}
\usepackage{amsfonts}
\usepackage{mathtools}
\usepackage[a4paper, portrait, margin=0.8in]{geometry}

\usepackage{listings}


\newcommand\perm[2][n]{\prescript{#1\mkern-2.5mu}{}P_{#2}}
\newcommand\comb[2][n]{\prescript{#1\mkern-0.5mu}{}C_{#2}}
\newcommand*{\field}[1]{\mathbb{#1}}

\DeclarePairedDelimiter\ceil{\lceil}{\rceil}
\DeclarePairedDelimiter\floor{\lfloor}{\rfloor}

\newcommand{\Mod}[1]{\ (\text{mod}\ #1)}

\begin{document}

\maketitle

\subsection*{}

\begin{enumerate}

%%%%% PROBLEM 1 %%%%%
\item You are given a deck of many, many cards. The all have values from $A=1$ to King$=13$. Cards with the same value are also ranked by suit. Spades $>$ Hearts $>$ Clubs $>$ Diamonds. Find an efficient way of sorting a given array of Cards (Card[]) in ascending order. (i.e. Ace of hearts should show up before Ace of spades, and Ace of spades should show up before three of spades)

The definition for the Cards class is as follows:

\begin{lstlisting}[language=Java]
 public class Card {
    int value; # From 1 to 13
    String suit; # Either "spade", "heart", "club" or "diamond".
 }
\end{lstlisting}

\item You are given a list of algorithm runtimes in $HH:MM:SS$ format. Return an array with these times in sorted order (i.e. $[04:05:11,10:15:01]$). You can assume you already have functions that can parse out the hours, minutes, and seconds portions of a time and can form a time string with properly padded 0's given the hours, minutes, and milliseconds. ($minutes("04:05:11")=5$, $getTime(4,5,11)="04:05:11"$)
    
\item You have the $X$ and $Y$ coordinates of $N$ points on the plane. You are also given the coordinates of point $A$. Return the $K$ closest points to $A$.

\item Write a function \texttt{mergeKLists(lists)}, where lists is an array of length $k$ of sorted linked lists. Merge the $k$ linked lists together into one sorted linked list, and return the head of the new linked list. For example, \texttt{lists} might look like (for k=3): % David has python solution in leetcode

\begin{lstlisting}
[
    1 -> 5 -> 8,
    3 -> 4,
    5 -> 5 -> 9
]
\end{lstlisting}

Which should return the linked list \texttt{[1 -> 3 -> 4 -> 5 -> 5 -> 5 -> 8 -> 9]}. What is the runtime and space of your algorithm in terms of $N$ (the average number of elements in each linked lists) and $k$ (the number of linked lists).
\end{enumerate}

\clearpage

\begin{lstlisting}[language=Java]
CARD_SORT(cards):
    suitMap = {diamond -> 0, club -> 1, heart -> 2, spade -> 3)
    for card in cards :
        card.overallValue = (card.value - 1) * 4 + suitMap[card.suit]
        # Cards now have a value from 0 to 51
    countingSort(cards, sort by cards.overallValue)
    
SORT_RUNTIMES(times):
    for time in times:
        time.totalSecs = seconds(time) + 60 * minutes(time) + 3600 * hours(time)
    minTime = min(times, in terms of times.totalSecs)
    
    # offset by minTime so we save space at the beginning of our array
    countingSort(cards, sort by times.totalSecs - minTime)

    
GET_CLOSEST_POINTS(points,A,k):
    for point in points:
        point.distance = (point.x - A.x)^2 + (point.y - A.y)^2
    pq = new MaxHeap<Point>(), use point.distance as key
    for point in points:
        pq.offer(point)
        if(pq.size > k) pq.poll()
    return pq.asArray
    
MERGE_K_LISTS(lists): # input has an array of linked lists
    if(lists.length == 1) return lists[0]
    lists += [] # useful for division by 2 kind of stuff
    for(i = 0; i < lists.length/2; i++):
        # merge second list into the first one
        MERGE(lists[i] + lists[i + lists.length/2])
    return MERGE_K_LISTS(lists[:lists.length/2])

\end{lstlisting}

% FIND_KTH(k, A, B, s1, s2) {
%     if (A or B is empty) {
%         return kth element in (B or A) ; // take note of starting indices s1/s2
%     }
    
%     if (k == 1) {
%         return min of A[s1] and B[s2];
%     }
    
%     midIndexA = s1 + k/2 - 1;
%     midIndexB = s2 + k/2 - 1;
    
%     midValueA = A[midIndexA] if within bounds else INF;
%     midValueB = B[midIndexB] if within bounds else INF;
    
%     if (midValueA < midValueB) {
%         return FIND_KTH(k-k/2, A, B, midIndexA+1, s2);
%     } else {
%         return FIND_KTH(k-k/2, A, B, s1, midIndexB+1);
%     }
    
% }



\end{document}

\grid
